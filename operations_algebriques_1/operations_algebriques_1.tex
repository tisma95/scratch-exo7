\documentclass[class=report,crop=false, 12pt]{standalone}
\usepackage[screen]{../scratch}

\begin{document}

\titre[F]{Opérations algébriques I}
%===============================

On représente les calculs par des arbres :

\myfigure{0.6}{
\tikzinput{arbre1}
}

Par exemple, l'arbre de gauche représente l'opération $2 + 3$, alors que l'arbre de droite représente l'opération $5 - 4$.

Pour un arbre plus grand, on effectue les opérations en partant du haut.

\myfigure{0.6}{
\tikzinput{arbre2}
}

Par exemple, pour effectuer le calcul de l'arbre de gauche, on commence par faire le calcul de $5 \times 4$, ce qui donne l'arbre de droite. Il reste alors à calculer $20 + 3$. L'arbre de droite représente donc le calcul $5 \times 4 + 3$. Ainsi le résultat est $23$.

\bigskip
\bigskip

\begin{activite}
\sauteligne 
\begin{enumerate}
  \item Effectue les calculs suivants (si possible de tête).

\myfigure{0.6}{
\tikzinput{arbre3}
}

  
  \item Représente sous forme d'un arbre les expressions suivantes (et calcule le résultat).
  
  $$
  12 \times 7 + 9
  \qquad
  12 + (3 - 5)
  \qquad
  8 \times (7 + 5)
  \qquad
  8 \times 7 + 8 \times 5
  \qquad
  (6 \times 8) \div (13 - 9)
  $$
 
\end{enumerate}

\end{activite}


\begin{activite}[niveau 5e et plus]
\sauteligne
\begin{enumerate}
  \item Effectue les calculs suivants.
  
\myfigure{0.6}{
\tikzinput{arbre4}
}

  
  \item Écris pour chaque arbre l'expression algébrique correspondante, puis développe-la.

\myfigure{0.6}{
\tikzinput{arbre5}
}

  
  \item Écris pour chaque arbre l'expression algébrique correspondante, puis développe-la.
  
\myfigure{0.6}{
\tikzinput{arbre6}
}
 
\end{enumerate}
\end{activite}


\begin{activite}[niveau 4e et plus]
\sauteligne
\begin{enumerate}
  \item Effectue les calculs suivants.

\myfigure{0.6}{
\tikzinput{arbre7}
}
  
  \item Écris pour chacune des expressions algébriques l'arbre correspondant et effectue les calculs.
  
  $$
  \frac 74 + \frac 83
  \qquad
  \frac{7 + 8}{12 \times 5}
  \qquad
  \frac{ \frac 23 }{ \frac 74 }
  \qquad
  \frac{ 2 + \frac 27 }{ \frac 14 - 9 }
  $$

\end{enumerate}
\end{activite}


\begin{activite}[niveau 3e et plus]

On note $a \wedge 2$ pour $a \times a$, on note $a \wedge 3$ pour $a \times a \times a$, 
on note $a \wedge 4$ pour $a \times a \times a \times a$...

\begin{enumerate}
  \item Effectue les calculs suivants.
  
\myfigure{0.6}{
\tikzinput{arbre8}
}
  
  \item  Simplifie les expressions suivantes (exprimées sous forme d'arbre) à l'aide de la notation \og $\wedge$ \fg{}.

\myfigure{0.6}{
\tikzinput{arbre9}
}  

  \item 
  \begin{enumerate}
    \item Écris l'arbre de $(a+b)^2$ et de son développement.
    \item Écris l'arbre de $(a-b)^2$ et de son développement.
    \item Écris l'arbre de $(a+b)(a-b)$ et de son développement.
  \end{enumerate}  
\end{enumerate}
\end{activite}


\begin{activite}
\sauteligne
\begin{itemize}
  \item L'expression $x \leftarrow 2$, signifie que la variable $x$ prend la valeur $2$.
  \item Si, ensuite, on rencontre l'instruction $x \leftarrow x + 1$, cela signifie que la nouvelle valeur de $x$ est l'ancienne valeur de $x$ plus $1$. Comme ici $x$ valait d'abord $2$, alors après l'instruction $x \leftarrow x + 1$, la nouvelle valeur de $x$ est $3$.
  \item Si on exécute encore une fois l'instruction $x \leftarrow x + 1$, alors $x$ vaudra $4$.
\end{itemize}

\begin{enumerate}
  \item Calcule la valeur finale de $x$.
  \begin{enumerate}
    
    \item 
    \begin{itemize}
      \item $x \leftarrow 3$
      \item $x \leftarrow x - 1$
      \item $x \leftarrow x + 3$
    \end{itemize}
    
    \item
    \begin{itemize}
      \item $x \leftarrow 3$
      \item $x \leftarrow 3 \times x$ 
      \item $x \leftarrow x + 1$
    \end{itemize}
    
    \item 
    \begin{itemize}
      \item $x \leftarrow 3$
      \item $x \leftarrow x + 1$
      \item $x \leftarrow 3 \times x$
    \end{itemize}
    
    \item 
    \begin{itemize}
      \item $x \leftarrow 3$
      \item $x \leftarrow 7 - x$
      \item $x \leftarrow x \times x$
    \end{itemize}     
  \end{enumerate}    
  
  \item Recommence les calculs en partant de l'instruction $x \leftarrow 4$ (au lieu de $x \leftarrow 3$).
 
 
  \item Calcule la valeur de finale de $x$.
  
  \begin{enumerate}
    \item
    \begin{itemize}
      \item $a \leftarrow 5$
      \item $b \leftarrow 7$
      \item $x \leftarrow a + b$
      \item $x \leftarrow x + 1$
    \end{itemize}
    
    \item 
    \begin{itemize}
      \item $a \leftarrow 5$   
      \item $b \leftarrow 7$
      \item $x \leftarrow a \times b$
      \item $x \leftarrow x + a$
    \end{itemize}
    
    \item
    \begin{itemize}
      \item $a \leftarrow 5$   
      \item $b \leftarrow 7$
      \item $x \leftarrow a \times (2 \times b - a)$
      \item $x \leftarrow 3 \times x + b$
    \end{itemize}
    
  \end{enumerate}  
  
  \item Recommence les calculs en partant des instructions 
$a \leftarrow 4$ et $b \leftarrow 9$ (au lieu de $a \leftarrow 5$ et $b \leftarrow 7$).   
  
\end{enumerate}
 
\end{activite}


\begin{activite}
Tu as deux variables $a$ et $b$. Tu dois mettre le contenu de la variable $b$ dans la variable $a$ et celui de la variable $a$ dans la variable $b$.

Par exemple partant de $a \leftarrow 5$ et $b \leftarrow 7$, on veut qu'à la fin des instructions, la variable $a$ contienne $7$ et la variable $b$ contienne $5$. Bien sûr, la façon de procéder ne doit pas dépendre des valeurs initiales données à $a$ et $b$ (dans l'exemple $5$ et $7$).
% cela doit marcher avec toutes les valeurs initiales (pas seulement avec $5$ et $7$).

\begin{enumerate}
  \item Pourquoi la suite d'instructions suivantes ne convient-elle pas ?
\begin{itemize}
  \item $a \leftarrow 5$
  \item $b \leftarrow 7$
  \item $a \leftarrow b$ 
  \item $b \leftarrow a$ 
\end{itemize}
  
  \item Cherche une méthode qui fonctionne !
\end{enumerate}
\end{activite}

\end{document}
