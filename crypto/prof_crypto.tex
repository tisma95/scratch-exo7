\documentclass[class=report,crop=false, 12pt]{standalone}
\usepackage[screen]{../scratch}

\begin{document}

\titre[P]{Cryptographie}
%===============================



\section*{Objectifs}

\begin{itemize}
  \item Code de César : codage, décodage.
  \item Chiffre de Vigenère.
  \item Attaque statistique.
  \item Codage des caractères.
  \item Modulo.
\end{itemize}


\section*{Durée}

2 à 3 heures (??)

\section*{Les activités}

\begin{itemize}
  \item Le code de César est archi-classique et amusant. On peut même réaliser les doubles anneaux mobiles, maintenus par une attache parisienne.
  
  \item Un moyen rapide d'attaquer le code de César, à l'aide d'un ordinateur, est d'énumérer les 26 possibilités.
  
  \item Mais l'attaque statistique est beaucoup plus intelligente.
  
  \item On présente ici une partie du code \textsc{ascii}. On pourra s'amuser aussi avec l'unicode/utf8 et ses caractères issus de toutes les langues.
  Les fichiers textes ne stockent pas des caractères mais des entiers qui représentent un caractère.
  
  \item Les modulos sont peut-être un peu compliqués pour les collégiens, 
  Pour le code de César on n'est pas obligés de détailler tout, mais les modulos servent en fait à plein d'endroits (test de parité, test de divisibilité, récupérer le chiffre des unités, les deux derniers chiffres d'un entier,...)
  
  \item Le chiffre de Vigenère est une variante de César qui n'est plus sûre qu'en apparence. Pour un texte assez long (par rapport à la longueur du bloc) on fait une attaque statistique sur les premières lettres de chaque bloc, puis sur les secondes,... 
  
  \item La morale de cette fiche est que le code de César ou celui de Vigenère est une simple opération mathématique : l'addition. Pour décoder, il ne reste plus qu'à soustraire ! Toutes les techniques de cryptographie modernes sont basées sur les maths. 
\end{itemize}


\section*{Ressources}


\section*{People}

\begin{itemize}
  \item Auteur : Arnaud Bodin
\end{itemize}


\end{document}


