\documentclass[class=report,crop=false, 12pt]{standalone}
\usepackage[screen]{../scratch}


\begin{document}

% Commande spécifique

\definecolor{coul_prive}{rgb}{0.93,0.26,0}
\definecolor{coul_public}{rgb}{0.06,0.63,0}

\newcommand{\prive}[1]{\relax\ifmmode{\color{coul_prive} #1}\else{\bf\color{coul_prive} #1}\fi}
\newcommand{\public}[1]{\relax\ifmmode{\color{coul_public} #1}\else{\bf\color{coul_public} #1}\fi}


\titre[F]{Cryptographie}
%===============================



\begin{activite}[Le code de César]

Jules César transmettait ses messages de façon cachée. Par exemple, le message
\prive{ALLEZ ASTERIX} est transformé en \public{DOOHC DVWHULA}.
Chaque lettre est décalée de $3$ lettres : \prive{A} devient \public{D},
\prive{B} devient \public{E}, \prive{C} devient \public{F}\ldots \,
Lorsque l'on arrive à la fin de l'alphabet (\prive{W} devient \public{Z}), on repart du début :
 \prive{X} devient \public{A},
\prive{Y} devient \public{B} et \prive{Z} devient \public{C}.

La roue ci-dessous t'aide pour coder le message, une lettre sur l'anneau extérieur est codée en la lettre associée sur l'anneau intérieur.

\myfigure{0.75}{
\tikzinput{crypto-cesar-01}
}

Pour décoder un message, on recule de trois lettres : \public{D} se décode en \prive{A},
\public{E} se décode en \prive{B}...  Autrement dit, on passe de l'anneau intérieur à l'anneau extérieur.
 

\begin{enumerate}
  \item 
  \begin{enumerate}
    \item Code la phrase \prive{ABC DE L INFO DEPUIS ZERO}.
    \item Trouve le nom de la première femme informaticienne en décodant \public{DGD ORYHODFH}.
  \end{enumerate}
  
  \item Bien sûr, on peut changer le décalage.
  Avec un décalage de $10$ : \prive{A} devient \public{K}, \prive{B} devient \public{L}\ldots (roue de gauche). Avec un décalage de $13$ : \prive{A} devient \public{N}, \prive{B} devient \public{O}\ldots (roue de droite).
  
 \myfigure{0.7}{
\tikzinput{crypto-cesar-02}\qquad
\tikzinput{crypto-cesar-03}
}  
    
  \begin{enumerate}
    \item Pour un décalage de $10$ : code \prive{CHARLES BABBAGE} ; décode \public{WKMRSXO WOMKXSAEO}.
    \item Pour un décalage de $13$ : code \prive{ZEROS ET UNS} ; décode \public{TRBETR OBBYR}.
    Quelle est la particularité du codage et du décodage lorsque le décalage vaut $13$ ?
  \end{enumerate}  
 
  
\end{enumerate}

\end{activite}



\begin{activite}[Attaque du code de César]
On se place dans la peau d'un espion qui vient d'intercepter le message : 

\centerline{\public{NCAN XD WN YJB NCAN}}
mais qui ne connaît pas le décalage qui a servi à le coder.

Il existe seulement $26$ décalages possibles : le code de César n'est pas très sécurisé... C'est un peu long à tester pour un humain, mais très facile pour un ordinateur.

Il existe une autre façon d'attaquer le message codé, car une même lettre est toujours codée de la même façon. Par exemple, pour un décalage de $3$, la lettre \prive{A} devient toujours \public{D} et \prive{E} devient toujours \public{H}. Hors dans un long texte, les lettres n'apparaissent pas toutes avec la même fréquence.
Pour la langue française, les lettres les plus rencontrées sont dans l'ordre : \\
\centerline{E S A I N T R U L O D C P M V Q G F H B X J Y Z K W}
avec les fréquences :
\begin{center}
\begin{tabular}{|c|c|c|c|c|c|}
\hline
{E}&{S}&{A}&{I}&{N}&{T}\\
\hline
14,5\%&8\%&7,5\%&7\%&7\%&7\%\\
\hline
\end{tabular}
\end{center}

Si la lettre \prive{E} apparaît très souvent dans le message de départ, alors pour un décalage de $3$, la lettre \public{H} apparaîtra très souvent dans le message codé.

Voici donc la méthode pour décrypter un message :
\begin{itemize}
  \item On cherche la lettre la plus fréquente dans le message codé (imaginons que c'est le \public{K}).
  \item On suppose que cette lettre correspond au \prive{E}.
  \item On calcule le décalage (pour aller de \prive{E} à \public{K} c'est $6$).
  \item On essaie de décoder le message sur la base de ce décalage.
  \item Si on obtient un message incohérent, on recommence en supposant que la lettre la plus fréquente correspond à l'une des lettres suivantes \prive{S}, \prive{A}, \prive{I}, \prive{N}, \prive{T}.
\end{itemize}

Avec cette méthode, décrypte les messages suivants (chaque message a été codé avec un décalage différent) :

\begin{enumerate}
  \item \public{NCAN XD WN YJB NCAN}
  \item \public{DFC WPD PALFWPD OPD RPLYED}
  \item \public{QSMRW KVERH IX TPYW MRXIPPMKIRX}
\end{enumerate}

\end{activite}



\begin{activite}[Codage des caractères]

Les ordinateurs préfèrent les nombres aux lettres ! Chaque caractère est numéroté.
Voici la table \textsc{ascii} des premiers caractères.

\myfigure{0.8}{
\small
  \tikzinput{crypto-unicode-01}
} 

Par exemple, le caractère numéro $37$ est le symbole \og \% \fg{}.
Le caractère $65$ est la lettre majuscule \og A \fg{}. Le caractère $107$ est la lettre minuscule \og k \fg{}. Les numéros $0$ à $32$ ne sont pas des caractères imprimables.

\begin{enumerate}
  \item Quelles phrases sont codées par les nombres suivants ?
  \begin{enumerate}
    \item 66-111-110-106-111-117-114-33
   
  %  [\symbol{66}\symbol{111}\symbol{110}\symbol{106}\symbol{111}\symbol{117}\symbol{114}\symbol{33}]

    \item 50-43-51-61-53 
    
  %  [\symbol{50}\symbol{43}\symbol{51}\symbol{61}\symbol{53}]
    
    \item 83-101-114-118-105-114 \quad 101-110  \quad 115-105-108-101-110-99-101-46 (La devise de la \textsc{nsa}.)
   
 %   [\symbol{83}\symbol{101}\symbol{114}\symbol{118}\symbol{105}\symbol{114}\     \symbol{101}\symbol{110}\ \symbol{115}\symbol{105}\symbol{108}\symbol{101}\symbol{110}\symbol{99}\symbol{101}\symbol{46}]
  \end{enumerate}

  \item Trouve l'équivalent numérique des caractères des noms suivants :
  Boole, Godel, Turing.
  
  \item 
  \begin{itemize}
    \item On note \codeinline{chr} la fonction qui à un nombre associe le caractère correspondant.
  Par exemple \codeinline{chr(65)} renvoie le caractère \og A \fg{}.
    \item On note \codeinline{ord} la fonction qui à un caractère associe son numéro dans la table ci-dessus. Par exemple \codeinline{ord('a')} renvoie l'entier $97$.
  \end{itemize}
  \begin{enumerate}
    \item Que donnent les instructions suivantes :
    \codeinline{chr(100)}, \codeinline{ord('H')}, \codeinline{chr(65+10)}, 
     \codeinline{ord(chr(77))}, \codeinline{chr(ord('\#'))} ?
     
    \item  Que fait la suite d'instructions suivantes ? (Essaye d'abord avec la lettre 'a'.)     
    \begin{itemize}
      \item Entrée : un caractère, noté \codeinline{car}, parmi 'a', 'b',\ldots, 'z'
      \item \codeinline{n} $\leftarrow$ \codeinline{ord(car)}
      \item \codeinline{n} $\leftarrow$ \codeinline{n - 32}
      \item Sortie : \codeinline{chr(n)}
    \end{itemize}
    
  \end{enumerate}  
  
\end{enumerate}
\vspace*{-1ex}
\end{activite}


\begin{activite}[Modulo]
Compter modulo $n$, c'est compter uniquement avec les entiers $0$, $1$, $2$,\ldots, $n-1$.
\begin{itemize}
  \item \textbf{Exemple modulo $60$.} 
  Compter modulo $60$, c'est compter comme les minutes d'une montre : $0$, $1$, $2$,\ldots, $59$. Quand on arrive à $60$, on repart immédiatement à $0$.
On va noter : 
$$60 \pmod{60} = 0 \qquad 61 \pmod{60} = 1 \qquad  62 \pmod{60} = 2 \qquad \ldots $$

  \item \textbf{Exemple modulo $12$.} 
  Si on compte modulo $12$, alors on se ramène à un entier parmi $0$, $1$,\ldots, $11$.
  On peut s'aider d'une roue pour visualiser :
  $$16 \pmod{12} = 4 \qquad 29 \pmod{12} = 5 \qquad  34 \pmod{12} = 10$$
 
\myfigure{0.8}{
\small
  \tikzinput{crypto-modulo-01}
}  
  
\end{itemize}

\begin{enumerate}
  \item 
  \begin{enumerate}
    \item Calcule $21 \pmod{12}$ ; $32 \pmod{12}$ ; $50 \pmod{12}$ ; $100 \pmod{12}$.
    \item Calcule $75 \pmod{60}$ ; $128 \pmod{60}$ ; $666 \pmod{60}$.
    \item Calcule $32 \pmod{26}$ ; $42 \pmod{26}$ ; $111 \pmod{26}$.
  \end{enumerate}
  
  \item On calcule $a \pmod{n}$ comme le reste de la division euclidienne de $a$ par $n$.
  Par exemple pour calculer $136 \pmod{21}$, on écrit la division euclidienne de $136$ par $21$ : $136 =  6 \times 21 + 10$. Donc $136 \pmod{21} = 10$.

\myfigure{1}{
  \tikzinput{crypto-modulo-02}
}  

  \begin{enumerate}
    \item Calcule $1\,254 \pmod{12}$.
    \item Calcule $5\,678 \pmod{60}$.
    \item Calcule $32\,158 \pmod{26}$.
  \end{enumerate}  
  
%  \item L'addition et la multiplication sont compatibles avec les modulos.
%  Par exemple pour calculer $18+32 \pmod{12}$ il y a deux façon de faire :
%  \begin{itemize}
%    \item d'abord $18+32 = 50$, puis $50 \pmod{12} = 2$,
%    \item mais c'est plus efficace de calculer d'abord $18 \pmod{12} = 6$
%    et $32 \pmod{12} = 8$ et donc 
%    $$18 + 32 \pmod{12} = 6 + 8 \pmod {12} = 14 \pmod{12} = 2$$
%  \end{itemize}
  \begin{enumerate}
    \item \textbf{Modulo $2$.}
    
    Calcule $3 \pmod{2}$ ; $4 \pmod{2}$ ; $5 \pmod{2}$\ldots
    Complète et retiens les énoncés suivants :
    
    \mybox{$a \pmod{2} = 0$ lorsque $a$ est \ \ \underline{\qquad\qquad\qquad}}
    
    \mybox{$a \pmod{2} = 1$ lorsque $a$ est \ \ \underline{\qquad\qquad\qquad}}
           
    \item \textbf{Modulo $10$.}
    
    Calcule $21 \pmod{10}$ ; $39 \pmod{10}$ ; $2345 \pmod{10}$.
    Complète et retiens l'énoncé suivant :
    
    \mybox{$a \pmod{10}$ est le \ \ \underline{\qquad\qquad\qquad\qquad\qquad\qquad} de l'entier $a$.}
    
    \item \textbf{Modulo $n$.}
    
    Calcule $(12\times 7) \pmod{7}$ ; $66 \pmod{11}$ ; $72 \pmod{9}$.
    Complète et retiens :
    \mybox{$n$ divise $a$ exactement lorsque \ \ \underline{\qquad\qquad\qquad\qquad\qquad\qquad}}
    
    
   \end{enumerate}
   
   
   \item \textbf{Retour au code de César.}  
   
   Le code de César est en fait une simple addition ! Numérotons les lettres par leur rang :
   \og A \fg{} est de rang $0$, \og B \fg{} est de rang $1$,\ldots, \og Z \fg{} est de rang $25$.
   Le code de César de décalage $3$ consiste simplement à ajouter $3$ au rang de la lettre. Comme le rang de la lettre ne peut atteindre $26$, il faut calculer le résultat modulo $26$.
   La formule pour un décalage de $3$ est donc :
   $$\text{rang codé} = \text{rang} + 3 \pmod{26}$$
   Pour le code de César de décalage $k$, la formule est :
   \mybox{$\text{rang codé} = \text{rang} + k \pmod{26}$}

\myfigure{0.8}{
\footnotesize
  \tikzinput{crypto-modulo-03}
} 
  
   \begin{enumerate}
     \item Calcule le rang codé par un décalage de $8$ pour les lettres de rang $7$, $15$, $23$.
     \item Complète les lignes du tableau ci-dessous
     en suivant le modèle de la première ligne :
     la lettre \og C \fg{} est de rang $2$ ; avec un décalage de $3$, le rang codé est $2+3\pmod{26}=5$ ; la lettre codée est donc le \og F \fg{}. Attention ! le plus petit rang est $0$ (lettre \og A \fg{}).
     
\begin{center}
\begin{tabular}{|c|c|c|c|c|}
\hline
Lettre & Rang & Décalage $k$ & Rang codé & Lettre codée \\ \hline\hline
C & 2 & 3  & 5 & F \\ \hline
F &   & 3  &   &   \\ \hline
Y &   & 3  &   &   \\ \hline
H &   & 15 &   &   \\ \hline
  & 10&    &   & R \\ \hline
T &   &    & 2 &  \\ \hline
  &   & 10   &   & E \\ \hline  
\end{tabular}
\end{center}
     
  \end{enumerate}  
   
   \item \textbf{Algorithme du code de César.}
   
   \'Ecris un algorithme qui, pour un décalage $k$ fixé, prend en entrée une lettre majuscule et renvoie en sortie la lettre codée par le code de César de décalage $k$.
   Par exemple si $k=3$ et si la lettre entrée est \og A \fg{} alors la sortie doit être \og D \fg{}.
   Les étapes de l'algorithme sont :
   \begin{itemize} 
     \item prends le caractère en entrée et transforme-le en un entier compris entre $65$ et $90$ (voir l'activité précédente) ;
     \item transforme cet entier en un nombre entre $0$ et $25$ ;
     \item applique la formule du décalage de César sur cet entier ;
     \item revient à un entier entre $65$ et $90$, puis à un caractère.
   \end{itemize}    
   
   
\end{enumerate}

\end{activite}


\begin{activite}[Le chiffrement de Vigenère]
Le code de César n'est pas assez sûr, le chiffrement de Vigenère en est une version plus sophistiquée. Par exemple pour coder : 

\centerline{\prive{ALLEZ ASTERIX}}

\begin{itemize}
  \item on commence par découper notre texte en blocs de même longueur, par exemple de longueur $3$ :
  
\centerline{\prive{A\ L\ L \qquad E\ Z\ A \qquad S\ T\ E \qquad R\ I\ X}}
 
 \item on choisit une clé, composée de $3$ nombres, par exemple $(3,6,5)$ ;
 
 \item on décale de $3$ la première lettre de chaque bloc (le \prive{A} du premier bloc devient \public{D}) ; 
 
 \item on décale de $6$ la deuxième lettre de chaque bloc (le premier \prive{L} du premier bloc devient \public{R}) ;
 
 \item on décale de $5$ la troisième lettre de chaque bloc (le second \prive{L} du premier bloc devient \public{Q}) ;
 
 \item on obtient par blocs \public{DRQ\  HFF\  VZJ\ UOC} et le message codé est alors :
 
\centerline{\public{DRQHF FVZJUOC}}
\end{itemize}

\bigskip

Note les deux améliorations par rapport au code de César classique :
\begin{itemize}
  \item une même lettre peut être codée de plusieurs façons (par exemple le premier 
  \prive{L} est codé en  \public{R} alors que le second est codé en \public{Q}) ;
  
  \item une même lettre du message codé peut correspondre à différentes lettres du message initial (par exemple le premier \public{F} code un \prive{Z} alors que le second \public{F} code un \prive{A}).
\end{itemize}
 
\bigskip
 

\begin{enumerate}
  \item Code la phrase \prive{RIEN NE SE PERD} avec la clé $(3,6,5)$.
  
  \item Décode la phrase \public{WUZW YJ WXFQYKRXRH} codée avec la clé $(3,6,5)$.
  
  \item Code la phrase \prive{EQUATEUR} avec la clé $(1,25,10,5)$.
  
  \item Décode la phrase \public{NDBNEHOS}, codée avec la clé $(1,25,10,5)$.  
  
  \item Décode la phrase suivante d'Albert Einstein, par une attaque par fréquence, sachant qu'elle a été codée par un chiffrement de Vigenère avec une clé de longueur $3$ :
  
  \centerline{\public{NE CKI J GWA ESTOI BPI IKGFEPLVXL}}
   \centerline{\public{KP MCYA CZHPGLT TVWV UG THU TLTHYG P LSYPNMITI}}
  
\end{enumerate}

\end{activite}




\end{document}

