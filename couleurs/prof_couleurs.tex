\documentclass[class=report,crop=false, 12pt]{standalone}
\usepackage[screen]{../scratch}

\begin{document}

\titre[P]{Couleurs}
%===============================



\section*{Objectifs}

\begin{itemize}
  \item Présenter le codage RVB (\emph{RGB} en anglais).
  \item \'Ecriture hexadécimale.
\end{itemize}


\section*{Durée}

2 heures (??)

\section*{Les activités}

\begin{itemize}
  \item Le système RVB est le plus connu mais n'est pas le seul. C'est en fait assez délicat de décrire toutes les couleurs et les autres systèmes sont plus compliqués. 
  
  \item Un autre problème est que tout le monde ne perçoit pas les couleurs de la même façon. Autre problème : l'affichage des couleurs est très différents à l'écran/par un vidéo-projecteur/sur une feuille imprimée. 
  
  \item On distinguera l'addition des couleurs (la formule de l'activité), du mélange des couleurs.
  
  \item Il existe des formules intéressantes pour le mélange des couleurs, pour le passage d'une image couleur vers une image noir et blanc.
  
  \item La façon d'indiquer le ton d'une couleur n'est pas universelle (pourcentage, hexadécimal, entier décimal). C'est l'occasion de pratiquer de la règle de trois.
  
  \item C'est surtout l'occasion d'introduire l'écriture hexadécimale. Rappelons qu'il faut bien distinguer un nombre de l'écriture du nombre. La différence est fondamentale mais très difficile à assimiler pour les élèves plongés dans la représentation décimale depuis tout petit. On a bien $16 = 10000_{bin} = A0_{hex}$  
  : c'est le même nombre, mais avec des écritures différentes. 
  
 \item Les avantages de l'hexadécimal : (1) écriture plus compacte (2 symboles pour les entiers jusqu'à $255$) ; (2) passage facile vers le binaire (c'est l'occasion de rappeler que les ordinateurs travaillent avec les puissances de $2$).
 
 \item On retrouve l'hexadécimal : dans les couleurs, dans les adresses ip,...
 
 \item Exercices complémentaires possibles :  passage binaire/hexadécimal ;
 écriture des entiers $>255$ ; écriture dans d'autres bases...
 
 
 \end{itemize}



\section*{Ressources}


\section*{People}

\begin{itemize}
  \item Auteur : Arnaud Bodin
  \item Certaines figures viennent des forums Tikz.
\end{itemize}

\end{document}


