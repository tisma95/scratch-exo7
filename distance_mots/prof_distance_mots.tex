\documentclass[class=report,crop=false, 12pt]{standalone}
\usepackage[screen]{../scratch}

\begin{document}

\titre[P]{Distance entre deux mots}
%===============================



\section*{Objectifs}

\begin{itemize}
  \item Définir des distances entres des mots.
  \item Présenter un algorithme textuel.
\end{itemize}


\section*{Durée}

2 heures (??)

\section*{Les activités}

\begin{itemize}
  \item Pour la distance de Hamming on pourrait assez facilement prouver que c'est une vraie distance (au sens mathématique du terme).
  
  \item La distance de Jaccard est aussi un distance au sens mathématique du terme.
  
  \item La distance de Levensthein est aussi un distance au sens mathématique du terme. Elle est beaucoup plus difficile à calculer. Il serait bon de faire prendre conscience aux élèves du problème suivant : ce n'est pas parce que l'on a trouvé un chemin de longueur $4$ par exemple, que quelqu'un de plus malin n'arrivera pas à trouver un chemin de longueur $3$. Il est très difficile de prouver que l'on ne peut pas faire mieux.
  
  \item L'algorithme de Levensthein est une façon de justifier la valeur de ce minimum. Encore une fois c'est un peu à long formaliser mais cela s'explique en deux minutes à l'oral.
  
  \item La remontée de l'algorithme pour trouver les opérations pourra être réservée aux classes les plus motivées.
  
  
  \item Il existe des variantes ou l'on autorise l'échange des deux lettres adjacentes. Il paraît que $80\%$ de fautes de frappes sont de l'un des $4$ types : un caractère en moins ; un caractère en trop ; un mauvais caractère ; deux caractère adjacents échangés.
  
    
\end{itemize}



\section*{Ressources}

\section*{People}

\begin{itemize}
  \item Auteur : Arnaud Bodin
  \item Sur une idée originale d'Éric Wegrzynowski
\end{itemize}

\end{document}


