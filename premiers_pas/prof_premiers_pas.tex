\documentclass[class=report,crop=false, 12pt]{standalone}
\usepackage[screen]{../scratch}

\begin{document}

\titre[P]{Premiers pas}
%===============================



\section*{Objectifs}

\begin{itemize}
  \item Se préparer aux déplacements avec Scratch.
  \item Distinguer une direction relative et absolue.
  \item Notion d'instructions et de code.
\end{itemize}


\section*{Durée}

1 heure (??)

\section*{Les activités}

\begin{itemize}
  \item Apprendre à exécuter et lire des instructions.
  \item Comprendre que l'ordinateur n’exécute ni plus ni moins que les instructions demandées (même s'il s'agit d'aller dans un mur !).

  \item La première activité s'inspire d'un exercice de communication afin de justifier l'intérêt du \emph{feed back}. Dans l'expérience originale, on doit d'abord faire dessiner sans autoriser de questions, puis dans un deuxième temps en acceptant des questions. Les dessins du deuxième \emph{round} sont bien évidemment meilleurs. Ici, c'est le processus inverse. Lors de la troisième phase, on est proche des instructions à donner à un ordinateur, cela met bien en évidence le fait que les instructions doivent êtres précises et sans ambiguïté.  Par contre, par rapport à un ordinateur, le langage n'est pas limité à un nombre restreint d'instructions.   
  
  \item Notions de coordonnées.
  \item S'orienter par des directions absolues : N/S/E/O.
  \item Ou par des directions relatives : droite/gauche.
  \item On pourra faire prendre conscience aux élèves et à l'oral de cette différence relatif/absolu. Par exemple : le suivant/le premier, être plus grand que/être le plus grand, dans 2h/à 16h00...
  \item On retrouve en d'autres notions d'informatique cette différence relatif/absolu :
  \begin{itemize}
    \item dans un tableur le contenu des cellules dépend souvent des cellules voisines, c'est une gestion relative des emplacements, mais on peut aussi faire référence à une cellule fixe. 
    \item la gestion de mémoire dans les ordinateurs, on écrit un nouveau fichier dans l'emplacement libre suivant, par contre pour retrouver le fichier il faut conserver son emplacement absolu.    
  \end{itemize}
\end{itemize}



\section*{Ressources}

\section*{People}

\begin{itemize}
  \item Auteur : Arnaud Bodin
\end{itemize}


\end{document}


