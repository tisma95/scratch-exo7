\documentclass[class=report,crop=false, 12pt]{standalone}
\usepackage[screen]{../scratch}

\begin{document}

\titre[P]{Pixels}
%===============================



\section*{Objectifs}

\begin{itemize}
  \item Comprendre la structure élémentaire d'une image.
  \item Tracé d'un segment.
  \item Différents algorithmes pour un même objectif.
\end{itemize}


\section*{Durée}

2 heures (??)

\section*{Les activités}

\begin{itemize}
  \item La première activité a deux buts. (1) Un peu de culture générale sur la taille et le format des images ou des écrans. (2) Un peu de proportionnalité.
Cela devrait être assez frappant que des écrans ayant le même rapport d'image
se situent sur une même droite.

  \item Il n'est peut être pas facile de convaincre les élèves que ce n'est pas évident de tracé un segment sur un écran. Du fait de la discrétisation, il y a des pixels à choisir et le choix est en fait arbitraire. L'algorithme de Bresenham est un tel choix.
  
  \item On commence par définir une petite axiomatique de ce que doit être un "bon" segment. Toute la subtilité est dans la règle (d).
  
  \item L'algorithme avec les réels peut être sauté, c'est ce qui vient naturellement à l'esprit. C'est l'occasion de voir le passage d'un réel à un entier. On pourra présenter d'autres façons (troncature, arrondi vers le haut). 
Mais en tout cas manipuler des réels est une mauvaise solutions, les calculs étant énormément plus lents. (Malheureusement il semble difficile de quantifier à quel point les calculs sur les entiers sont plus rapides qu'avec les réels.) 
  
  \item Le véritable algorithme de Bresenham est le dernier, celui qui manipule uniquement des entiers.     
Il est en fait assez facile à manipuler, mais un peu longuet à expliquer par écrit. Le mieux est quand  même de l'expliquer aux élèves sur un exemple. 
Il y a deux entiers à pré-calculer, puis un entier qui varie.

 \item Exercices complémentaires possibles : cas des segments plus verticaux qu'horizontaux ; Bresenham pour les cercles.
 
 
 \end{itemize}



\section*{Ressources}


\section*{People}

\begin{itemize}
  \item Auteur : Arnaud Bodin
  \item 
\end{itemize}

\end{document}


