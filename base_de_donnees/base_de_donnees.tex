\documentclass[class=report,crop=false, 12pt]{standalone}
\usepackage[screen]{../scratch}


\begin{document}

% Commades spécifiques
\newcommand{\ducode}[1]{\textsf{#1}}


\titre[F]{Bases de données}
%===============================

\textbf{Première partie. La bibliothèque.}

Voici les tables qui contiennent les informations sur les livres, les lecteurs et les emprunts d'une bibliothèque.

\bigskip

{\footnotesize
% Livres
\begin{minipage}{0.7\textwidth}

\textbf{Table 1 : Livres}\\
\emph{Titre avec année de parution.} \\

\begin{tabular}{|l|l|l|} \hline
\textbf{Identifiant} & \textbf{Titre} & \textbf{Année} \\ \hline\hline
L1 & Harry Cover 1 & 2005 \\ \hline
L2 & Anselme Lupin, le cambrioleur & 1965 \\ \hline
L3 & L'énigme de la chambre verte & 1912 \\ \hline
L4 & Le voleur des anneaux 1 & 1984 \\ \hline
L5 & Harry Cover 2 & 2007 \\ \hline 
L6 & Convergente & 2014 \\ \hline
L7 & L'aiguille vide & 1956 \\ \hline
L8 & Le jeu de la fin & 2012 \\ \hline 
L9 & Bilboquet le hobbit & 1991 \\ \hline
L10 & Le voleur des anneaux 2 & 1987 \\ \hline 
\end{tabular}
\end{minipage}
%
%
% Auteurs
\begin{minipage}{0.25\textwidth}

\textbf{Table 2 : Auteurs}\\
\emph{Nom et année de naissance.} \\

\begin{tabular}{|l|l|l|} \hline
\textbf{Identifiant} & \textbf{Nom} & \textbf{Année} \\ \hline\hline
A1 & Lenoir & 1905 \\ \hline
A2 & Rolling & 1974 \\ \hline
A3 & Levert & 1912 \\ \hline
A4 & Colline & 1990 \\ \hline
A5 & Trollquin & 1942 \\ \hline
A6 & Rosse & 1990 \\ \hline 
\end{tabular}
\end{minipage}
}
%
%

\bigskip

{\footnotesize
% Emprunteurs
\begin{minipage}{0.4\textwidth}

\textbf{Table 3 : Emprunteurs}\\
\emph{Prénom et année de naissance.} \\

\begin{tabular}{|l|l|l|} \hline
\textbf{Identifiant} & \textbf{Prénom} & \textbf{Année} \\ \hline\hline
E1 & Amandine & 2005 \\ \hline
E2 & Valentin & 2003 \\ \hline
E3 & Victor & 2007 \\ \hline
E4 & Sonia & 2006 \\ \hline
E5 & Benjamin & 2006 \\ \hline
E6 & Junior & 2001 \\ \hline
E7 & Clémence &  2003 \\ \hline
E8 & Nadia & 2003 \\ \hline
\end{tabular}
\end{minipage}
%
%
% Table Livres-Auteurs
\begin{minipage}{0.25\textwidth}

\textbf{Table 4 : Livres--Auteurs}\\

\begin{tabular}{|l|l|} \hline
\textbf{Id. livre} & \textbf{Id. auteur}  \\ \hline\hline
L2 & A1 \\ \hline
L4 & A5\\ \hline
L1 & A2 \\ \hline
L8 & A6 \\ \hline
L7 & A1 \\ \hline
L10 & A5 \\ \hline
L8 & A4 \\ \hline
L6 & A4 \\ \hline
L3 & A3 \\ \hline
L5 & A2 \\ \hline
L6 & A6 \\ \hline
L9 & A5 \\ \hline 
\end{tabular}
\end{minipage}
%
%
% Table Livres-Emprunteurs
\begin{minipage}{0.25\textwidth}

\textbf{Table 5 : Livres--Emprunteurs}\\

\begin{tabular}{|l|l|} \hline
\textbf{Id. livre} & \textbf{Id. emprunteur}  \\ \hline\hline
L4 & E1 \\ \hline
L5 & E4 \\ \hline
L8 & E7 \\ \hline
L7 & E3 \\ \hline
L10 & E7 \\ \hline
L1 & E4 \\ \hline
L6 & E2 \\ \hline
L9 & E1 \\ \hline 
\end{tabular}
\end{minipage}
}


\begin{activite}
\emph{Dans cet exercice on utilise seulement les tables 1, 2 et 3.}

\begin{enumerate}
   \item Quel est le titre du livre paru en dernier ?
   \item Quels sont les livres parus à 100 ans d'écart ?
   \item Combien de livres ont été publiés après 1980 ?
   \item Quel est l'identifiant de l'auteur né juste après Trollquin ?   
   \item Quel est le prénom de l'emprunteur le plus jeune.    
   \item Quels sont les prénoms des emprunteurs nés après 2004, classés par ordre alphabétique ?
   \item Quels sont les emprunteurs filles, classées par âge croissant ?
   
\end{enumerate}

\end{activite}


\begin{activite}

\emph{Dans cet exercice on utilise aussi les tables 4 et 5.}


\medskip

\begin{itemize}
  \item Les données de la première ligne de la table 4 signifient que le livre \og L2 \fg{} a pour auteur \og A1 \fg{}.
  Un livre peut avoir plusieurs auteurs et un auteur peut écrire plusieurs livres.
 
  \item Les données de la première ligne de la table 5 signifient que le livre \og L4 \fg{} est emprunté par \og E1 \fg{}.
  Un livre ne peut être emprunté que par une personne, mais il est possible d'emprunter plusieurs livres !
\end{itemize}


 \begin{enumerate}
  \item Quel est le nom de l'auteur du livre \og L7 \fg{} ?
  \item Quel est le titre du livre écrit par \og Levert\fg{} ?
  \item Combien de livres a écrit \og Trollquin \fg{} ?
  \item Quels livres ont plusieurs auteurs ? 
  
  \item Quels livres a empruntés Amandine (classés par ordre ordre alphabétique du titre) ? 
  \item Quels livres ne sont pas empruntés (classés par date de parution) ?
  \item Qui est fan d'Harry Cover ?
  
  \item Qui a emprunté le livre le plus récent ?
  
  \item Classe les emprunteurs, en commençant par celui qui a le plus de livres, si deux personnes ont emprunté le même nombre de livres, commence par le plus jeune.
  
\end{enumerate}

\end{activite}



\begin{activite}

On utilise un langage spécifique pour obtenir des réponses à partir des tables.
Par exemple la requête :

  \indentation\ducode{sélectionner \textbf{Titre} dans la table \textbf{Livres}}

renvoie toute la colonne \emph{Titre} de la table \emph{Livres}.
  
  
 \begin{enumerate}
  \item  Écris le résultat de la requête :
  
  \indentation\ducode{sélectionner \textbf{Nom} dans la table \textbf{Auteurs}}
    
  \item Écris le résultat de la requête dans laquelle on impose une condition :
  
  \indentation\ducode{sélectionner \textbf{Prénom} dans la table \textbf{Emprunteurs}}\\
  \indentation\indentation\ducode{avec \textbf{Année} $\ge 2005$}
   
  \item Écris le résultat de la requête dans laquelle on demande de trier les résultats :
  
  \indentation\ducode{sélectionner \textbf{Titre} dans la table \textbf{Livres}}\\
  \indentation\indentation\ducode{avec \textbf{Année} $\le 1980$}\\
  \indentation\indentation\ducode{trier par ordre alphabétique}    
  
  
\end{enumerate}

\end{activite}


\bigskip
\bigskip
\textbf{Seconde partie. Le cinéma.}


Voici les tables qui contiennent les informations sur l'activité d'un cinéma : films, salles, horaires et séances.

\bigskip

{\footnotesize
% Films
\begin{minipage}{0.6\textwidth}

\textbf{Table 1 : Films}\\
\emph{Titre et durée (en minutes).} \\

\begin{tabular}{|l|l|l|} \hline
\textbf{Identifiant} & \textbf{Titre} & \textbf{Durée} \\ \hline\hline
F1 & L'homme scarabée & 120 \\ \hline
F2 & La guerre des planètes & 90 \\ \hline
F3 & Superfemme & 100 \\ \hline
F4 & Le retour du Jedaï & 120 \\ \hline
F5 & La vengeance du Jedaï & 100 \\ \hline
F6 & Le monde préhistorique perdu & 110 \\ \hline
F7 & Bateau contre iceberg & 130 \\ \hline
F8 & Rapide et pas content & 100 \\ \hline
F9 & L'homme fer à repasser & 90 \\ \hline
F10 & Jacques Bon & 90 \\ \hline
\end{tabular}
\end{minipage}
%
%
% Salles
\begin{minipage}{0.4\textwidth}

\textbf{Table 2 : Salles}\\
\emph{Salle avec le nombre de personnes qu'elle peut accueillir.} \\

\begin{tabular}{|l|l|} \hline
\textbf{Identifiant} & \textbf{Capacité} \\ \hline\hline
S1 & 210 \\ \hline
S2 & 180 \\ \hline
S3 & 170 \\ \hline
S4 & 200 \\ \hline
S5 & 210 \\ \hline
S6 & 180 \\ \hline
\end{tabular}
\end{minipage}
}
%

\bigskip

%
{\footnotesize
% Livres
\begin{minipage}{0.4\textwidth}

\textbf{Table 3 : Créneaux}\\
\emph{Jour et heure d'ouverture.} \\

\begin{tabular}{|l|l|l|} \hline
\textbf{Identifiant} & \textbf{Jour} & \textbf{Heure} \\ \hline\hline
C1 & Mardi & 18h00 \\ \hline
C2 & Mardi & 21h00 \\ \hline
C3 & Mercredi & 14h00 \\ \hline
C4 & Mercredi & 19h00 \\ \hline
C5 & Samedi & 19h00 \\ \hline
C6 & Samedi & 20h00 \\ \hline
C7 & Dimanche & 11h00 \\ \hline
C8 & Dimanche & 18h00 \\ \hline
C9 & Dimanche & 21h00 \\ \hline
\end{tabular}
\end{minipage}
%
%
% Livres
\begin{minipage}{0.4\textwidth}

\textbf{Table 4 : Séances}\\
\emph{Une séance est définie par un créneau, une salle et un film} \\

\begin{tabular}{|l|l|l|} \hline
\textbf{Id. créneau} & \textbf{Id. salle} & \textbf{Id. film} \\ \hline\hline
C1 & S3 & F3 \\ \hline
C2 & S6 & F7 \\ \hline
C2 & S1 & F3 \\ \hline
C3 & S3 & F1 \\ \hline
C4 & S5 & F6 \\ \hline
C4 & S2 & F7 \\ \hline
C5 & S5 & F2 \\ \hline
C6 & S1 & F4 \\ \hline
C6 & S5 & F10 \\ \hline
C7 & S2 & F3 \\ \hline
C8 & S6 & F7 \\ \hline
C8 & S3 & F3 \\ \hline
C9 & S5 & F4 \\ \hline
\end{tabular}
\end{minipage}
}

\bigskip

Voici, par exemple, ce que signifient les données de la première ligne de la table 4 :
le mardi à 18h00 (créneau C1) dans la salle 3 (salle S3) est projeté le film \og Superfemme \fg{} (film F3).


\bigskip

\begin{activite}
À l'aide des tables ci-dessus, réponds aux questions suivantes :
 \begin{enumerate}
  \item Quels jours passe le film \og Bateau contre iceberg \fg{} ?
  \item Quels sont les films projetés le mercredi ?
  \item Quel est le film le plus long passant à 19h00 ?
  \item Quelle salle n'est jamais utilisée ?
  \item Quel film est projeté le plus souvent ?
  \item Quels films sont projetés le samedi et le dimanche (classés par ordre alphabétique du titre) ?
  \item Quels films ne sont pas projetés (classés par ordre alphabétique inverse de leur titre) ?
  
\end{enumerate}

\end{activite}



\begin{activite}
\sauteligne

\begin{enumerate}
  \item  Écris le résultat de la requête qui renvoie deux colonnes :
  
  \indentation\ducode{sélectionner \textbf{Jour} et \textbf{Heure} dans la table \textbf{Créneaux}} \\
      \indentation\indentation\ducode{avec \textbf{Heure} = 21h00}
          
          
  \item Écris le résultat de la requête qui renvoie deux colonnes :
  
  \indentation\ducode{sélectionner \textbf{Identifiant} et \textbf{Capacité} dans la table \textbf{Salles}}  \\
  \indentation\indentation\ducode{trier par ordre croissant de \textbf{Capacité}}  
    
   
  \item Écris une requête (ainsi que le résultat) qui permet de sélectionner les films et leur durée, classés par durée croissante.
  
  \item Écris une requête (ainsi que le résultat) qui permet de sélectionner l'identifiant et la capacité des salles pouvant accueillir au moins 200 personnes.
    
\end{enumerate}

\end{activite}



\begin{activite}
\sauteligne

\begin{enumerate}
  \item Complète la table 4 pour ajouter une séance : le film \og La vengeance du Jedaï \fg{} dans la salle 6 dimanche à 11h00.
  
  \item Complète la table 4 pour ajouter une séance : le film \og Le monde préhistorique perdu \fg{} dans la salle 3 mardi à 21h00.  
  
  \item Complète la table 1 et la table 4 pour ajouter la projection du film 
  \og Avatariendutou \fg{} d'une durée de 1h40 le samedi à 20h00 en salle 7.
  
\end{enumerate}

\end{activite}

\end{document}


