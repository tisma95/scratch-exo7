\documentclass[class=report,crop=false, 12pt]{standalone}
\usepackage[screen]{../scratch}

\begin{document}

\titre[P]{Puissances de 2}
%===============================



\section*{Objectifs}

\begin{itemize}
  \item Manipuler les puissances de $2$.
  \item Quantité de mémoire.
\end{itemize}


\section*{Durée}

1 heure (??)

\section*{Les activités}

\begin{itemize}
  \item Encore une fois, lorsque l'on a deux états possibles, les puissances de $2$ apparaissent naturellement.
  
  \item Les élèves doivent connaître et reconnaître les puissances de $2$, au moins jusqu'à $2^{10}$.
  
  \item Il faut familiariser les élèves aux ordre de grandeurs des quantités de mémoire, leur faire prendre conscience du gigantisme (= Go) de la mémoire que nécessite une vidéo par rapport à un texte.
  
  \item Concernant la compression d'un fichier, on pourra aller plus loin : chercher les méthodes de compressions et les taux réels. Il faudra bien distinguer les types de compressions qui ne sont pas tous équivalents : la compression sans perte d'information (par exemple pour compresser un texte) ou avec perte d'information (pour une image ou une vidéo, on accepte que la vidéo compressée soit de moins bonne qualité).
  
  \item L’ambiguïté du kilo-octet est toujours dans les usages, même si depuis 1998, la norme internationale est que $1$ kilo-octet représente $1000$ octets. Beaucoup utilisent encore abusivement $1024$ octets. Le mot kibioctet est rarement utilisé, mais cependant vous trouverez parfois les notations Kio, Mio, Gio.
\end{itemize}


\section*{Ressources}


\section*{People}

\begin{itemize}
  \item Auteur : Arnaud Bodin
\end{itemize}


\end{document}


