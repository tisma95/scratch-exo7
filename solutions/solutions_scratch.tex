\documentclass[class=report,crop=false, 12pt]{standalone}
\usepackage[screen]{../scratch}

\begin{document}


\titre[E]{Solutions -- \'Enigmes Scratch}
%===============================


%------------------------------------
\section*{1 -- Premiers pas}

\begin{enigme}
Réponse : 350 pas.
\end{enigme}

\begin{enigme}
Réponse : $480$ quelle que soit la longueur des segments inconnus.
\end{enigme}

\begin{enigme}
Réponse : à l'étape numéro 12.
\end{enigme}


%------------------------------------
\section*{2 -- Répéter}

\setcounter{enigme}{0}

\begin{enigme}
Réponse : $x=150$, $y = 100$.
\end{enigme}

\begin{enigme}
Réponse : $360/15 = 24$ côtés.
\end{enigme}

\begin{enigme}
Réponse : $4 \times 3 = 12$ triangles.
\end{enigme}


%------------------------------------
\section*{3 -- Coordonnées $x,y$}

\setcounter{enigme}{0}

\begin{enigme}
Réponse : le nombre est $26$.
\end{enigme}

\begin{enigme}
Réponse : l'abscisse est environ $69$ ; les réponses valides sont donc $67,68,69,70,71$.
\end{enigme}

\begin{enigme}
Réponse : le nom à trouver est \mot{TURING}.
\end{enigme}

%------------------------------------
\section*{4 -- Si ... alors ...}

\setcounter{enigme}{0}

\begin{enigme}
Réponse : $46$ ou $47$ car $x=46,20$.
\end{enigme}

\begin{enigme}
Réponse : $167$ ou $168$ car $x=167,72$.
\end{enigme}

\begin{enigme}
Réponse : $80$ car les assertions 1 et 3 sont vraies, donc Scratch avance de $30+50=80$.
\end{enigme}


%------------------------------------
\section*{5 -- Entrée/Sortie}

\setcounter{enigme}{0}

\begin{enigme}
Réponse : $1645$, date de création de la pascaline.
\end{enigme}

\begin{enigme}
Réponse : \og \mot{python} \fg{} : un langage informatique moderne et puissant.
\end{enigme}

\begin{enigme}
Réponse : \og \mot{basic} \fg{} : un ancien langage informatique facile à apprendre.
\end{enigme}


%------------------------------------
\section*{6 -- Variables et hasard}

\setcounter{enigme}{0}

\begin{enigme}
Réponse : 2000. En effet : probabilité que la somme soit $5$ : $4/36$ ; probabilité que la somme soit $9$ : $4/36$.
Donc probabilité $5$ ou $9$ est $8/36 = 2/9 \simeq 0,22$.
Donc pour $10 \, 000$ lancers, environ $2200$ devraient être comptés !
\end{enigme}

\begin{enigme}
Réponse : $93$.
\end{enigme}

\begin{enigme}
Réponse : $31$ car le rapport est proche $10 \pi$.
\end{enigme}


%------------------------------------
\section*{7 -- Si ... alors ... sinon ...}

\setcounter{enigme}{0}

\begin{enigme}
Réponse : $184$.
\end{enigme}

\begin{enigme}
Réponse : $y = 21$.
\end{enigme}

\begin{enigme}
Réponse : $104$ car $x = 104$ et $y = 92$.
\end{enigme}


%------------------------------------
\section*{8 -- Plusieurs lutins}

\setcounter{enigme}{0}

\begin{enigme}
Réponse : de $24$ à $32$ car $x= 28$.
\end{enigme}

\begin{enigme}
Réponse : $144$. Les termes de la suite de Fibonacci sont : $1,2,3,5,8,13,21,34,55,89,144$.
\end{enigme}

\begin{enigme}
Réponse : $25$. 
\end{enigme}


%------------------------------------
\section*{9 -- Sons}

\setcounter{enigme}{0}

\begin{enigme}
Réponse : $41$.
\end{enigme}

\begin{enigme}
Réponse : de $50$ à $54$ car le pic se situe près de $52$.
\end{enigme}

\begin{enigme}
Réponse : 34.
\end{enigme}


%------------------------------------
\section*{10 -- Invasion}

Pas d'énigmes !


%------------------------------------
\section*{11 -- Créer ses blocs}

\setcounter{enigme}{0}

\begin{enigme}
Réponse : 3241
\end{enigme}

\begin{enigme}
Réponse : $n=3$, car \codeinline{monbloc(n)} dessine un polygone à $n$ côtés.
\end{enigme}

\begin{enigme}
Réponse : $n=7$.
\end{enigme}


%------------------------------------
\section*{12 -- Listes}

\setcounter{enigme}{0}

\begin{enigme}
Réponse : $180$ car $x_G = 80$ et $y_G = 100$. 
\end{enigme}

\begin{enigme}
Réponse : \og ruojnoB \fg{}. Les lettres sont inversées. 
\end{enigme}

\begin{enigme}
Réponse : environ $1300$.
\end{enigme}


\end{document}

