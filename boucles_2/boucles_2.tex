\documentclass[class=report,crop=false, 12pt]{standalone}
\usepackage[screen]{../scratch}


\begin{document}

\titre[F]{Boucles II}
%===============================

\emph{Le but de cette feuille est d'apprendre à concevoir la structure d'un programme sur feuille avant de se jeter sur le clavier !}

\bigskip
\bigskip

\begin{activite}
\sauteligne
\begin{enumerate} 
  \item Écris un programme de compte à rebours : demande une valeur $n$ (par exemple $10$), puis affiche la liste
  de décompte jusqu'à $0$ (par exemple $10$, $9$, $8$,\ldots, $0$).
  
  \item Écris un algorithme qui cherche le plus grand entier $n$ tel que $n \times n \le  20\,000$.
 
  \item Trouve un algorithme qui renvoie le plus petit entier $n$ tel que $2^n \ge 1\,000\,000$. (Tu peux par exemple initialiser une variable à $2$ et la multiplier par $2$ autant de fois que nécessaire.)
  
  \item Écris la suite d'instructions qui teste si un nombre est premier : demande un entier $n$ et utilise un test \og est-ce que $i$ divise $n$ ? \fg{}.
  On rappelle qu'un entier $n$ est premier s'il n'a pas de diviseurs autres que $1$ et $n$.
  
\end{enumerate}

\end{activite}


\bigskip


Une boucle \emph{pour} permet de parcourir un par un tous les éléments d'une liste.
Voici un exemple :
\begin{center}
\begin{minipage}{0.4\textwidth}
Pour $i$ allant de $1$ à $10$, faire :\\
\indentation afficher $i \times i$.
\end{minipage}
\end{center}
La variable $i$ va prendre successivement les valeurs $1$, puis $2$, puis $3$,... jusqu'à $10$.
Ce petit programme affiche les $i\times i$, c'est-à-dire dans cet ordre $1$, puis $4$, puis $9$,... jusqu'à $10 \times 10 = 100$. 


La syntaxe générale est (pour $a$ et $b$ entiers positifs ($a<b$)):
\begin{center}
\begin{minipage}{0.4\textwidth}
Pour $i$ allant de $a$ à $b$, faire :\\
\indentation instruction,\\
\indentation autre instruction,\\
\indentation ...\\
\end{minipage}
\end{center}
L'entier $i$ va successivement prendre la valeur $a$, puis $a+1$,... jusqu'à l'entier $b$.

\bigskip

\begin{activite}
\sauteligne
\begin{enumerate}
  \item Construis une boucle qui affiche les produits $2\times x$, $3\times x$,..., $20 \times x$ (où $x$ est un nombre à demander à l'utilisateur).
     
  \item Construis une boucle qui calcule la somme $1\times 1 \times 1 + 2\times 2 \times 2 + 3\times 3 \times 3 + \cdots
%  + i \times i \times i + \cdots 
+ n \times n \times n$ (où $n$ est un entier à demander à l'utilisateur).  
  
  \item Demande $10$ nombres à l'utilisateur et affiche la position du plus grand de ces nombres.
  Par exemple, si les nombres sont $2,3,5,10,2,1,3,3,1,5$ alors le plus grand nombre est $10$ et le programme renvoie la valeur $4$ (car $10$ est en quatrième position). 
  
  \item Construis un programme qui affiche tous les résultats de la table classique des multiplications
  (on affiche tous les produits $i \times j$, $i$ et $j$ étant des entiers allant de $1$ à $10$).
  
\end{enumerate}

\end{activite}


\begin{activite}
Les algorithmes suivants ne font pas ce que l'on attend d'eux. Trouve les problèmes et corrige-les !

\begin{enumerate}
  \item \textbf{But :} Le programme tire des cartes au hasard, il s'arrête lorsque la carte tirée est la dame de c\oe ur ou le roi de c\oe ur.
    
    \textbf{Solution fausse :}
 \myfigure{1}{
\footnotesize\tikzinput{boucles2_1}
} 
 
  \item \textbf{But :} Le programme affiche la table de multiplication par $7$ (c'est-à-dire les multiples de $7$ inférieurs ou égaux à $70$).
  
    \textbf{Solution fausse :}  
\myfigure{1}{
\footnotesize\tikzinput{boucles2_2}
} 
  
  \item \textbf{But :} Le programme affiche les nombres pairs compris entre $2$ et $100$ : $2,4,6,8,\ldots, 100$.

    \textbf{Solution fausse :}
\myfigure{1}{
\footnotesize\tikzinput{boucles2_3}
}  
   
  \item \textbf{But :} Le programme calcule le produit $10 \times 9 \times 8 \times \cdots \times 2 \times 1$.
  
      \textbf{Solution fausse :}
\begin{center}
\begin{minipage}{0.4\textwidth}
$P \leftarrow 1$\\
$n \leftarrow 10$\\
Tant que $P \ge 1$, faire :\\
\indentation $P \leftarrow P \times n$\\
\indentation $n \leftarrow n-1$\\
Sortie : $n$
\end{minipage}
\end{center}    
  
  \item \textbf{But :} Une somme, au départ de $1000$ €, rapporte chaque année $10$ \% d'intérêts (la somme d'argent est donc multipliée par $1,10$ chaque année). On veut savoir au bout de combien d'années la somme placée dépasse $2000$ €.
  
     \textbf{Solution fausse :} 
\begin{center}
\begin{minipage}{0.4\textwidth}
$S \leftarrow 1000$\\
$n \leftarrow 0$\\
Tant que $S \ge 2000$, faire :\\
\indentation $S \leftarrow S \times 1,10$\\
\indentation $n \leftarrow n+1$\\
Sortie : $n-1$
\end{minipage}
\end{center}  
  
\end{enumerate}
\end{activite}

\end{document}


  